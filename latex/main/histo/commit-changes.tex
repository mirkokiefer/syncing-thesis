
\section{Cross-Branch Change Detection}
\label{main.diff-across-commits}

The change detection phase of our synchronization protocol is defined through the calls `getCommitDifference()' and `getDataDifference()'.
We will now look into the internals of the algorithms invoked through these calls.\\

\subsection{Commit History Difference}
Identifying added commit IDs since a last known synchronized commit is only working on the meta-data level - there is no application data involved in this phase.\\
Given two branches A and B we want to retrieve all commits A needs in order to be in sync with B.
Our algorithm is based on the recursive invocation of a lowest common ancestor implementation:\\

\begin{enumerate}
\item Compute the lowest common ancestor of commit A and B.
\item Add commit B to the result.
\item Walk up the ancestor chain of commit B, adding all commits to the result unless:
\item The common ancestor is reached - then return the result.
\item If a commit has multiple ancestors - then invoke the algorithm again with each ancestor as commit B.\\
Add the result of the recursive invocation to the final result.
\end{enumerate}

We can express this more concretely using pseudo-code:\\

\begin{lstlisting}[caption=Detecting commit history difference, label=commit-difference]

function getCommitDifference(commitA, commitB) {
  result = []

  commonAncestor = getCommonAncestor(commitA, commitB)

  while (not commitB.hasMultipleAncestors()) {
    singleAncestor = commitB.getAncestors()[0]

    if (singleAncestor == commonAncestor) {
      return result
    }

    result.push(singleAncestor)

    commitB = singleAncestor
  }

  ancestors = commitB.getAncestors()

  for (each ancestor in ancestors) {
    forkResult = getCommitDifference(commitA, commitB)
    result.append(forkResult)
  }
}

\end{lstlisting}

Figure \ref{fig:histo.new-commits} visualizes the commit difference for two branches B and C.
The last synchronized commit is A.
B has in the meantime made concurrent changes and it has synchronized with another branch which is at commit D.
Branch D has not been synchronized with branch C.
The commit difference between A and B is therefore the sum of two commit paths:\\

\begin{itemize}
\item The shortest commit path between B and the lowest common ancestor of A and B, which is B.
\item The shortest commit path between D and the lowest common ancestor of D and A, which is E.
\end{itemize}

\begin{figure}[new-commits]
  \centering
  \includegraphics[width=0.6\textwidth]{img/new-commits}
  \caption{Commit difference between branches B and C and A being the last synchronized commit.}
  \label{fig:histo.new-commits}
\end{figure}

\subsection{Application Data Change Detection}

just use commits and identify added blobs for each by parallel tree walking of commit and ancestor...