% This template was initially provided by Dulip Withanage.
% Modifications for the database systems research group
% were made by Conny Junghans and Jannik Strötgen.

\documentclass[
     12pt,         % font size
     a4paper,      % paper format
     BCOR10mm,     % binding correction
     DIV14,        % stripe size for margin calculation
     liststotoc,   % table listing in toc
     bibtotoc,     % bibliography in toc
     idxtotoc,     % index in toc
%     parskip       % paragraph skip instad of paragraph indent
     ]{scrreprt}

%%%%%%%%%%%%%%%%%%%%%%%%%%%%%%%%%%%%%%%%%%%%%%%%%%%%%%%%%%%%

% PACKAGES:

% Use German :
\usepackage[USenglish]{babel}
% Input encoding
\usepackage[utf8]{inputenc}
% Font encoding
\usepackage[T1]{fontenc}
% Index-generation
\usepackage{makeidx}
% Einbinden von URLs:
\usepackage{url}
% Special \LaTex symbols (e.g. \BibTeX):
\usepackage{doc}
% Include Graphic-files:
\usepackage{graphicx}
% Include doc++ generated tex-files:
%\usepackage{docxx}
% Include PDF links
%\usepackage[pdftex, bookmarks=true]{hyperref}

% Fuer anderthalbzeiligen Textsatz
\usepackage{setspace}

\usepackage{listings}
\usepackage{color}

\definecolor{green}{rgb}{0,0.6,0}
\definecolor{gray}{rgb}{0.5,0.5,0.5}
\definecolor{mauve}{rgb}{0.58,0,0.82}

\lstset{
  belowcaptionskip=1\baselineskip,
  captionpos=b,
  breaklines=true,
  frame=L,
  xleftmargin=\parindent,
  showstringspaces=false,
  basicstyle=\footnotesize\ttfamily,
  commentstyle=\itshape\color{gray},
  numbers=left
}

% hyperrefs in the documents
\usepackage[bookmarks=true,colorlinks,pdfpagelabels,pdfstartview = FitH,bookmarksopen = true,bookmarksnumbered = true,linkcolor = black,plainpages = false,hypertexnames = false,citecolor = black,urlcolor=black]{hyperref} 
%\usepackage{hyperref}

% force graphics to be in same section as declared
\usepackage[section]{placeins}

%%%%%%%%%%%%%%%%%%%%%%%%%%%%%%%%%%%%%%%%%%%%%%%%%%%%%%%%%%%%

% OTHER SETTINGS:

% Pagestyle:
\pagestyle{headings}

% Choose language
%\selectlanguage{english}
%\newcommand{\setlang}[0]{\selectlanguage{USenglish}\nonfrenchspacing}


\begin{document}

% TITLE:
\pagenumbering{roman} 
\begin{titlepage}


\vspace*{1cm}
\begin{center}
\vspace*{3cm}
\textbf{ 
\Large Ruprecht Karls University Heidelberg\\
\smallskip
\Large Institute of Computer Science\\
\smallskip
\Large Database Systems Research Group\\
\smallskip
}

\vspace{3cm}

\textbf{\large Bachelor Thesis} % Studienarbeit, Interdisziplinaeres Projekt

\vspace{0.5\baselineskip}
{\huge
Offline Usage and Synchronization in Mobile Apps with HTML5
}
\end{center}

\vfill 

{\large
\begin{tabular}[l]{ll}
Name: & Mirko Kiefer\\
%Matricle: & 2746040\\
%Supervisor: & Prof. Dr. Gertz\\
%Date of Submission: & \today
\end{tabular}
}

\end{titlepage}

\onehalfspacing

\thispagestyle{empty}

\vspace*{100pt}
I declare that this thesis was composed by myself and that the work contained therein is my
own, except where explicitly stated otherwise in the text.

\vspace*{50pt}


Date of Submission: \today
\newpage

% Add a brief summary of your topic and contributions (Zusammenfassung) in German and in English:
\chapter*{Abstract}

% This file contains an abstract of your thesis, with approximaltely 300-500 words


\newpage

\chapter*{Zusammenfassung}

% This file contains the German version of your abstract, with about 300-500 words


\newpage

% MAIN PART:
% Table of contents (Inhaltsveryeichnis)
\tableofcontents
\cleardoublepage
\pagenumbering{arabic} 

% List of figures (Abbildungsverzeichnis):
%\listoffigures
% List of tables (Tabellenverzeichnis):
%\listoftables

%%%%%%%%%%%%%%%%%%%%%%%%%%%%%%%%%%%%%%%%%%%%%%%%%%%%%%%%%%%%%%%
% Here, the actual content of your thesis begins
% You can either put all the text here or use individual files to store the chapters of your thesis.
% Below are templates for both alternatives.


\chapter{Introduction}
\label{sec:intro}

\section{Motivation}
Applications that allow users to collaborate on data on a central server are in widespread use.
Popular examples are document authoring tools like Google Docs, project collaboration apps like Basecamp or Trello or even large scale collaboration projects like Wikipedia.\\

The traditional architecture of collaborative applications follows a client-server model where the server hosts the entire application logic and persistence.
Users access the application through a thin client, most commonly a web browser.
The browser only has to display user interfaces that are pre-rendered by the server.\\
This model works well when using desktop computers with a realiable, high-speed connection to the server.\\

Rising expectations on the user experience drove developers to increasingly move application logic to the client.
Initially this has only been the logic required to render user interfaces.
The server still hosted most of the application logic to pre-compute all relevant data for the client.\\
Moving the interface rendering to the client reduces the amount of data that has to be transferred and makes the application behave more responsive.\\

The widespread adoption of mobile devices forces developers to re-think their architecture again.
Users can now carry their devices with them and expect their applications to work outside their home or office network.
Applications therefore have to work with limited mobile Internet access or often no access at all.\\
The only way to support this is by moving more of the application logic to the client and by replicating data for offline use.
The clients are now not only responsible for rendering interfaces but also implement most of the application logic themselves.\\
The new architecture comes at a high price - the additional client logic and persistence adds a lot of complexity.
While in the server-centric model developers only had to maintain a single technology set, they now face different technologies on each platform they aim to support with a fat client.\\
The ability to use the application offline requires an entire new layer of application logic to manage the propagation and merging of changes and to resolve conflicts.
The only responsiblity of the server in this model is the propagation of data between clients.\\

Most users today carry a notebook, a smartphone and maybe even a tablet computer with them.
They often want to work with the same data on different devices.
Apps need to support workflows like adding some items to a Todo-Manager on a notebook and subsequently reviewing them on a smartphone.
This implies that even simple applications that are meant for single-users have to aquire collaborative features.
A single-user with multiple devices is from a technical perspective effectively collaborating with itself.\\
Today's applications only achieve this through  data synchronization between the devices and a central server.
If the user is mobile and does not have a reliable Internet connection he is stuck with outdated data on his smartphone.
This problem can only be resolved by supporting the direct synchronization between devices.
The clients can now basically act as servers themselves and manage propagation of data to other clients.\\
The actual server does not have to disappear in this model.
But like the clients it is just another node on the network.
The difference is that the server node is continuously connected to the Internet and can therefore play a useful role as a fallback.\\
Note that this only describes the most extreme scenario - in most real-world applications we will see a hybrid-architecture where clients can synchronize most data directly but the server still manages security or enforces other constraints.\\
Building such a distributed data synchronization engine including all relevant aspects is very complex and beyond the reach of a small team of app developers.
It is also way beyond the scope of this thesis.
As described in the next section we will focus on a set of problem statements and use cases.

TODO:

- add Things app story on how hard it is

- add graphics

\section{Objectives and Approach}

This thesis aims to develop patterns and tools to make the development of offline capable, collaborative apps more productive.\\

The guiding questions are:
\begin{itemize}
\item \emph{Offline Availability}: How can we enable the operation of a collaborative app with frequent network partition?
\item \emph{Synchronization Protocol}: How can we efficiently synchronize changed data directly between unreliably connected devices?
\item \emph{Application Integration}: How can we abstract the synchronization logic to be as unintrusive as possible to an application?\\
\end{itemize}

A collaborative app that has to function with unreliable network connection implies that we can not rely on the traditional thin client model.
We have to think about ways to make both data and logic available offline.\\

Being able to synchronize data directly between devices forces us to develop a distributed architecture.\\

Efficient synchronization means that we aim to minimize the amount of redundant data sent between devices.
We have to figure out ways to identify changes in the data.\\

Combined with the requirement to be unintrusive we exclude solutions that require the application to explicitly track changes in the code.
The identification of data changes should be decoupled from the main application logic.
This ensures that an upgrade of traditional applications requires minimal effort.\\

We will refine this set of requirements by breaking down common use cases and evaluating existing solutions that support offline-capable applications.\\

Important questions which are out of scope of this thesis are:

\begin{itemize}
\item \emph{Security}: How can we manage access rights and encryption in a distributed architecture?
\item \emph{Device Discovery}: How can we discover devices in a network to collaborate with?
\item \emph{Data Transmission}: How is the data propagated among devices on a technical level?
\end{itemize}

TODO:

- phrase problem statements - not only questions

- what is unique to our approach?

\section{Structure of the thesis}
Here you describe the structure of the thesis. For example:

In Kapitel~\ref{background} werden grundlegende Methoden für diese Arbeit vorgestellt.


\chapter{Background}
\label{sec:background}
We will start this chapter by explaining the core aspects of data synchronization.
After setting it in context with traditional properties of distributed databases we will present some of the popular approaches to synchronization.\\
Some technical background on the practicality of local data storage based on HTML5 will follow.\\
This will give us a solid background to develop and reason about our own data synchronization framework.

\section{Defining Data Synchronization}
Let us introduce some basic terminology and try to systematically define what data synchronization is about.

\emph{Atoms} are what we define as the lowest level of data that cannot be devided into smaller parts.
Every application may have a custom definition of atoms.
For a file synchronizer it may be entire files, for a source code management system it may be lines in a file, for a collaborative task manager it may be literal values like strings, numbers or dates.\\
Atoms can be aggregated to \emph{objects}.
A source code management system may define objects as a sequence of lines aggregated to a file.
The task manager could aggregate values like strings and dates to task objects.
Objects can themselves be aggregated further into larger objects.
File objects can be combined to directory objects, task objects into larger structures like a project.
Aggregation of objects can also be seens as relationships between objects.\\
A collaborative application has multiple users working on different devices on a related set of data.
They are either connected directly or via servers who live on the local network or the Internet.
Each device, be it a user's device or a server, we define as a \emph{node}.
Nodes can be connected through various network topologies like peer-to-peer, client-server or a hierarchical architecture.
In section \ref{sec:main.requirements.topologies} we will go into more detail about different network topologies.\\
The nodes of mobile users are likely to be partitioned from their network and therefore have to be able to work in \emph{offline} mode.
Therefore application data has to be available locally so that users are not blocked from using their application.
Even when connected to a network it can be beneficial to maintain data locally to increase the responseness of the application.
In a collaborative application, edits that are made locally will have to be \emph{synchronized} with other participating nodes.\\
The process of synchronization can be divided into three phases.
Local edits first have to be identified before they can be sent to other nodes.
We refer to this step as the \emph{update detection} phase.
Some applications may explicitly track each edit as its made and store the history of edit operations.
This \emph{edit-based} approach is necessary for \emph{stream-based synchronization} which we explain in section \ref{sec:background.stream-based}.\\
If edits are not tracked directly we have to run a differencing algorithm to detect updates.
This requires us to keep previous states of the data and is detailed further in section \ref{sec:background.history-based} on \emph{history-based synchronization}.\\
Once updates are detected we continue with the \emph{update propagation} phase.
A stream of edit operations or the differencing output is sent to the collaborating nodes.
The details will be explained in the respective sections on the stream or history based approaches.\\
In a final phase the received data has to be \emph{reconceiled} with the local data on each node.
Updates have to be merged and conflicts are identified.
In a centralized scenario this part is usually carried out by the server.
Distributed architectures supporting peer-to-peer synchronization are much more complex as all clients have to reconcile the received updates in an eventually consistent way.\\

Summarizing these terms:

\begin{itemize}
\item \textbf{Atoms} are the literal values that can not be divided further.
\item \textbf{Objects} aggregate atoms or other objects into larger structures.
\item \textbf{Nodes} are the collaborating devices in an application.
\item The \textbf{update detection} phase identifies local data changes on each node.
\item During \textbf{update propagation} changed data is sent to collaborating nodes.
\item \textbf{Reconciliation} merges data received from other nodes and identifies conflicts.
\end{itemize}

\emph{Update detection}, \emph{update propagation} and \emph{update reconciliation} combined are what we define as \emph{data synchronization}.

\section{ACID Properties and Transactions}
The nodes of a collaborative, mobile application with replicated data represent a distributed database.
Distributed database systems have been a focus of research for decades.
Traditionally, the incentive to make databases distributed has been to provide fault tolerance, increase read/write throughput or to increase storage capacities.
Mobile applications need data replication to reduce frequent network access resulting in a better user experience.
Traditional distributed databases used to back enterprise applications running entirely in server farms.
Servers are connected through reliable and high-speed networks.
Network partitions are the absolute exception.\\
On mobile devices network partitions or slow connections are the norm.
Users want to work with their notebooks even when not being in an office environment with reliable Internet access.
Mobile networks are still comparatively slow and unreliable.\\
Back in 1981 Jim Gray defined the properties of a reliable transaction system \cite{Gray:1981wi}.
They are referred to as the ACID (Atomicity, Consistency, Isolation, Durability) properties - a term coined by Andreas Reuter and Theo Härder \cite{Reuter1983}.
Consistency is defined as the property ensuring that a database can only transition between valid states.
One way to achieve this is to use locking so that a record can not be edited concurrently.
In an always-connected server environment transactions are measured in seconds - locking of data can therefore be acceptable.
In a mobile setting this is not an option as transactions can easily last days.
A mobile user who wants to edit some data while travelling without network connection should certainly not block all other users from doing their work.
Using locking concepts in such a scenario would not only be a an inconvenience for the users but would actually lead to a high-rate of deadlocks.
As Jim Gray states, the rate of deadlocks goes up with the square of the level of concurrency and the fourth power of the transaction size \cite{Gray3}.\\
A common alternative to locking is \emph{multiversion concurrency-control} where readers can still access the prior version of data being editd by another user.
Distributed databases usually use a two-phase commit protocol to gurantee strong consistency.
Each participant has to agree in order to successfully complete a transaction.
In a mobile setting with long periods of disconnection each commit could take hours or days to be acknowledged by all nodes.
This is further complicated as nodes are often not fixed and can be added or removed from a mobile application at any time.


In a distributed database this property can only 

TODO:

- master-less database, no global coordinater

- offline by default --> no globally serializable commits, two-phase commit no suitable

- atomicity (all or nothing commit) only on local nodes

- only eventual consistency

- isolation: has to be relaxed as we cannot allow locking --> concurrent transactions cannot be serialized, use lowest level of isolation ("Read uncommitted", concurrent commits to local databases will eventually be consistent

- durability: only local durability, global durability can not be guaranteed in a master-less database

\section{Stream-Based Synchronization}
\label{sec:background.stream-based}
An application that tracks each edit and sends it in a stream to remote nodes follows a stream-based synchronization protocol.
Stream-based synchronization is very common among real-time document editors like Google Docs.\\

An edit usually represents an insert or delete operation at a certain position in the text.
These edit operations are broadcast to remote nodes and then ``replayed''.
As participating nodes can concurrently edit a document the stream of edit operations can not just be applied without modifications.\\
The combination of local modifications and received edit operations from a remote node requires the transformation of the remote operations in order to be correctly applied.\\
The family of algorithms developed to correctly transform the edit operations is described as \emph{Operational Transformation} \cite{Ellis:1998vf}.\\
If some nodes are temporarily offline while continuing to edit, the correct transformation of many concurrent edit operations becomes very complex and error-prone.\\
A practical problem in modern user interfaces is that it is hard to correctly capture all edits made to data.
If a single edit is missed the result is a fork possibly rendering all future update operations as incorrect.
Packet loss due to unreliable network connections have also be taken into account which further complicates the design of a robust algorithm.\\
Research has therefore investigated options for data synchronization that do not require Operational Transformation.\\

\emph{Commutative Replicated Data Types} (CRDTs) have emerged as a viable alternative for specific use cases.
A recent study by Shapiro et al. presents a range of data types designed for synchronization without concurrency control \cite{Shapiro:2011wy}.\\
CRDTs are designed in a way that all edit operations commute when applied in \emph{causal order}.
Section \ref{sec:main.requirements.causality} goes into more detail about causal ordering of events.
Due to the restrictions on supported operations on data types, CRDTs are only applicable in a narrow set of scenarios.

TODO:

- more details about OT and CRDTs

\section{History-Based Synchronization}
\label{sec:background.history-based}
Snapshot-based methods work by tracking and relating an application's data state over time.
Instead of sending a sequential stream of raw updates, each client collects additional metadata that allows more complex reasoning about the state of each client.\\
A prominent example is the distributed content tracking system \emph{git} \cite{git} which can resolve the most complex peer-to-peer synchronization scenarios.\\
Git achieves this by storing the entire history of a project's database on each client.
Each edit made to objects in the database is stored as a commit object and related to its ancestors.\\
Through the resulting commit graph each client can identify the exact subset of updates each remote node has to receive in order to be in sync.\\
While it sounds extremely inefficient to store the entire history of a database, git manages to do this in a very efficient way through a \emph{Content Addressable Store} and data compression.
It is not uncommon that the uncompressed form of the current state of a git project is larger than the project's entire history.

\section{Three-Way Merging}
\label{sec:background.merging}
Three-way merging describes the concept for an algorithm that performs a merge operation on two objects based on a common ancestor.\\
Let \emph{A} be the initial state of the object and let \emph{B} and \emph{C} be edited versions of \emph{A}.
The goal is to merge \emph{B} and \emph{C} into a new object \emph{D}.\\
The merge algorithm starts by identifying the differences between \emph{A} and \emph{B} and between \emph{A} and \emph{C}.\\
All \emph{parts} of object \emph{B} that are neither changed in \emph{B} nor in \emph{C} are carried over into \emph{D}.\\
All changes to parts of the object in \emph{B} that have not been changed in \emph{C} are directly accepted and added to \emph{D}.\\
If the same parts are edited both in \emph{B} and \emph{C} we have a merge conflict that needs to be resolved.\\
There is no universal algorithm for resolving conflicts.
Different types of data and applications require different types of conflict resolution strategies.
In many cases conflict resolution can not even be done in an automated way but has to be left to the user of an application.\\
Even the term \emph{three-way merging} only describes a general concept but the actual algorithm will differ based on the type of objects that are merged.
Text files are the most common type of object with lines seen as the \emph{parts}.
The unix program \emph{diff3} implements a three-way merge variant for text files \cite{diff3}.\\
Most modern version control systems implement three-way merging to allow lock-free collaboration on source code.
\emph{Git} applies three-way merging not only for text files but for entire file system trees \cite{git}.\\
With git we have a great example of a hierarchical conflict resolution strategy:

\begin{itemize}
\item If two developers concurrently edit the same directory git tries to resolve this conflict by descending into the directory and looking at individual files.
\item If the developers edited different files git can automatically resolve the conflict by accepting both changes.
\item If the same file was edited concurrently git tries to descend a level deeper by looking at edits made to individual lines.
\item If different lines were edited concurrently it can again resolve the conflict by accepting both changes.
\item Only in the unlikely event that both developers edited the same line git has no way to automatically resolve the conflict.
It will delegate the conflict resolution to the developers who will have to manually merge both changes.
\end{itemize}

Tancred Lindholm designed a three-way merging algorithm for XML-documents.
With the \emph{3DM} tool there is even an implementation available \cite{Lindholm:2001uv}.
As XML supports the expression of a broad range of data types this is probably one of the most generic implementations.

\section{Most Recent Common Ancestor}
\label{background.mrca}
- describe problem with graphs

- describe solution referring to standard algo

\section{Content Adressable Storage}
- copy on write

- simple verification of data, free checksums

- git as example

\section{HTML5 and Offline Applications}
HTML5 specifies a number of client-side storage options. Most are a work in process and still have to be adopted by all browser vendors. IndexedDB is most likely going to be the standard for building offline-capable web applications. Combined with Cache Manifests, HTML5 provides all the tools necessary for building offline applications.

\subsection{Web Storage}
The simplest API is the \emph{localStorage} standard defined in the W3C's Web Storage specification \cite{webstorage}.\\
It provides a key-value store accessible from JavaScript which can store string values for string keys.
Most browsers currently set a storage limit of 5 MB per site.
\emph{LocalStorage} is therefore only suitable for storing small volumes of data.\\
Another limitation is the interface which is synchronous. As JavaScript is single-threaded, every read or write operation will block the entire application.
Frequent or large-volume read/write operations can result in a bad user experience caused by a ``freezing'' user-interface.\\
\emph{LocalStorage} is currently supported by all major browsers including its mobile variants.

\subsection{Web SQL Database}
A much more advanced implementation is specified by the now deprecated \emph{Web SQL} standard \cite{websql}. It defines a relational database similar to Sqlite including SQL support.\\
The proposal was strongly opposed by the Mozilla Foundation who sees a SQL-based database as a bad fit for web applications \cite{mozilla_indexeddb}.\\
The standard was therefore only implemented by Google Chrome, Safari and Opera and their mobile counterparts in Android and iOS.\\
\emph{Web SQL} has been officially deprecated by the W3C and support by browsers is likely going to drop in the future.

\subsection{Indexed Database}
Instead of Web SQL the standard favored by the W3C and most browser vendors is \emph{IndexedDB}.\\
\emph{IndexedDB} defines a lower-level interface for storing key/value pairs and setting up custom indexes.
While relatively simple, the API design is generic enough to cater for implementations of more complex databases on top.
It would, for example, be possible to implement a \emph{Web SQL} database using \emph{IndexedDB}.\\
IndexedDB supports storing large amounts of data and defines an asynchronous API.\\
Unfortunately the standard has not yet been implemented across all major browsers.
It is currently available in Mozilla Firefox, Google Chrome and Internet Explorer.
Safari support is still missing as well as support in the default Android and iOS browser.\\
Luckily most browsers who have not implemented IndexedDB yet, are still supporting Web SQL.
There is a polyfill available that implements an IndexedDB interface using Web SQL \cite{indexeddb_polyfill}. Application developers can therefore already base their work on the IndexedDB interface while browser vendors are catching up.

\subsection{Cache Manifests}
To truely work offline, an application has to make its static resources available locally as well.
The \emph{cache manifest} defined in the HTML standard gives developers the right tool \cite{cache_manifests}. It allows you to define a local cache of all application resources like HTML, CSS, JavaScript code or other static files.\\
Flexible policies give fine-grained control over which resources should be available offline and which need network connection.



\chapter{Designing a Synchronization Framework}\label{main}

\section{Application Scenarios}
We describe common synchronization scenarios based on popular mobile applications.

\subsection{A Collaborative Task Manager}

The seemingly simple scenario of a Task Manager turns out to be very complex to implement if enhanced with collaborative features and offline availability.

Let us first look at some user stories and then try to define a suitable data model for such an application.

\subsubsection{User Story 1: Creating Projects}
\begin{itemize}
\item A User can create Projects in order to coordinate Tasks.
\item A User can invite other Users as Project Members to a Project.
\end{itemize}

Examples for Projects created by User Rita would be:\\

\begin{tabular}{ l l }
Project Name & Members \\
\hline
Marketing Material & Rita, Tom, Allen \\
Product Roadmap & Rita, Allen \\
Sales Review & Rita, Lisa
\end{tabular}

\subsubsection{User Story 2: Creating and Editing Tasks}
\begin{itemize}
\item Project Members can add Tasks to a Project in order to manage responsibilities.
\item A Task can have a due date and responsible project member assigned.
\item A Task can be edited by Project Members and marked as done.
\item A Task can be moved in the list of Tasks.
\end{itemize}

An example list of Tasks could be:\\

\begin{tabular}{ l l l l }
\multicolumn{4}{ c }{Project "Marketing Material"} \\
Task & Due Date & Assignee & Done \\
\hline
Create event poster & 2013-08-12 & Rita & No\\
Write blog entry on event & 2013-07-20 & Tom & Yes
\end{tabular}

\subsubsection{User Story 3: Commenting on Tasks}
\begin{itemize}
\item Project Members can add Comments to Tasks
\end{itemize}

Examples would be:\\

\begin{tabular}{ l l l }
\multicolumn{3}{ c }{Task "Create event poster" in Project "Marketing Material"} \\
Member & Date & Comment \\
\hline
Rita & 2013-07-20 & Allen, I need you to create some graphics. \\
Allen & 2014-07-20 & Ok, lets go through it tomorrow morning!
\end{tabular}

\subsubsection{User Story 3: User Workflows}
\begin{itemize}
\item In order to be productive a user needs to access all Tasks from any device.
\item A user should be able to Edit and Create Projects and Tasks when disconnected from any network.
\item The data should be kept as current as possible even if a user's device does not have reliable internet access.
\end{itemize}

An example workflow that should be supported:\\

\begin{itemize}
\item Rita works at the desktop computer in her office with high-speed internet access. She creates Project A and invites Allen.
\item Allen works from home on his notebook with high-speed internet access. He reviews the Project and creates task A1.
\item Rita is already on her way home but has mobile internet access on her smartphone. She receives the added task A1 and edits its title.
\item Rita is still on the train but decides to continue working on her notebook. Her notebook does not have internet access but she can establish a direct connection to her smartphone via Wifi. The reception on her smartphone has dropped in the meanwhile. She receives the latest updates from her smartphone and adds a comment to task A1.
\item Allen who is still at home can not receive Rita's comment as she is still on the train. In the meanwhile he creates a Task A2 in Project A.
\item Rite gets home where she has internet access with her notebook. She receives Allen's created Task A2.
\item Allen, who is still at his notebook, receives Rita's comment as soon as she connects to internet at home.
\end{itemize}

\subsubsection{Data Model}


\begin{itemize}
\item User (name, email, has Todo Lists)
\item Invited User (name, email)
\item Invited User List (has Invited Users)
\item Todo Item (title, description, due date, belongs to Todo List)
\item Todo List (name, belongs to Users, has Todo Items)
\end{itemize}

The User type has a singleton instance who represents the user of the
app.\\Users can be invited to Todo Lists. As their list of Todo Lists is
hidden from the current user Invited User is a separate type.\\Invited
User List is simply a cached list of all users that have been invited in
the past.\\While Invited User List is an unordered list, Todo Lists and
Todo Items are ordered.

Syncing lists of unordered object IDs never causes conflicts while
syncing ordered object IDs can cause order conflicts.

\subsection{A File Sharing App}

Dropbox synchronizes a file system - it is therefore a good example for
syncing of hierarchical data.

The data model is simple:

\begin{itemize}
\item Tree Item (name)
\item Tree extends Tree Item (has children of type Tree Items)
\item Data extends Tree Item (data)
\end{itemize}

The list of child Tree Items can either be ordered or unordered. While
Dropbox does not sync the order of files there are scenarios where this
is required.

Syncing trees can trigger conflicts if sub trees have been modified
concurrently.

\subsection{(Text Synchronization)}

Collaborative document editors like Google Docs need to synchronize text
that is concurrently edited.\\Google Docs currently does not support
offline editing.

Syncing text is equal to the problem of syncing an ordered list and can
trigger conflicts.

\section{Requirements}
\label{sec:requirements}
From the common scenarios we derive a set of requirements for a synchronization solution.

Requirements for strategies:

\begin{itemize}
\item Causality preservation
\item Eventual consistency
\item Optimistic synchronization
\item Expose conflicts
\item Support peer-to-peer or hybrid synchronization
\item Integration with existing app logic
\end{itemize}

(TODO: need to explain why this set of requirements, constraints on
mobile devices\ldots{})

Aspects to consider when evaluating strategies:

\begin{itemize}
\item How are updates detected?
\item How are updates propagated? (Stream or Snapshot)
\item How are updates merged/reconciled? (State or Edit-based)
\item Level of structural awareness (Textual, Syntactic, Semantic/Structural)
\end{itemize}

\section{Evaluating Existing Systems}
Here we evaluate existing solutions based on the requirements.

\subsection{git}

\begin{itemize}
\item
  Data structure: filesystem/tree
\item
  Merging: tree-based, three-way merge
\item
  Propagation: snapshot-based
\item
  Supports peer-to-peer
\end{itemize}

\subsection{CouchDB}

\begin{itemize}
\item
  Data structure: key-value
\item
  Merging: tree-based
\item
  Propagation: stream-based
\item
  Supports peer-to-peer
\end{itemize}

\subsection{(Backends-as-a-Service)}
- parse.com
- stackmob
- deployd

Most of them simply expose a REST-API but leave all the conflict handling work to the app developer.\\
Are completely centralized.

\section{Architecture of Synclib}
Based on the requirements and the evaluation of existing systems we derive a unique architecture for a practical synchronization solution.

\begin{itemize}
\item
  \textbf{no timestamps}: state-based 3-way merging
\item
  \textbf{no change tracing}: change tracing is not necessary - support
  diff computation on the fly
\item
  \textbf{data agnostic}: leave diff and merge of the actual data to
  plugins
\item
  \textbf{distributed}: syncing does not require a central server
\item
  \textbf{be small}: only implement the functional parts of syncing -
  leave everything else to the application (transport, persistence)
\item
  \textbf{sensitive defaults}: have defaults that \emph{just work} but
  still support custom logic (e.g.~for conflict resolution)
\end{itemize}

- cross-platform through web standards
- solve server behaviour through native proxy
- diff-merge-patch
- most-recent-common-ancestor

\section{Technologies used for Implementation}
We describe implementation details like the technologies used, code structure and the testing framework to evaluate the system.

- everything web-based --> only way to be cross-platform
- client-side persistence with HTML5
- note on alternatives (Lua, native)

\section{Differencing and Merging of Data Models}
- explain diff, merge and patch
- implement diff, merge and patch logic for primitive data structures
  -> use them to recursively model complex data structures
- ensure conflicts are made explicit

\subsection{Sets}
\subsection{Ordered Lists}
\subsection{Ordered Sets}
\subsection{Dictionaries}
used for object collections in data models
\subsection{Ordered Dictionaries}
most common for managing ordered object collections in data models
can be modeled with dictionary and ordered set/list
\subsection{Trees}
- tree as an example for composite data model
- efficient child tree pointers like in git

\subsection{Composite Data Structures}
- show how to represent complex data models as composite data structures

\section{Storing and Commiting Changes}
As syncing is state based we need to track the history of edits on each client.\\Each client has his own replica of the database and commits
data locally.\\On every commit we create a commit object that links both
to the new version of the data and the previous commit.\\

- use content-adressable store
- only store changes and reference unchanged data through hashs --> like git
- commit links to data and parent commit

\section{Finding Common Commits}
- Most Recent Common Ancestor algorithm used for finding common commit of clients
- described algorithm in background
- implementation as separate module

\section{Synchronization Protocol}
If a client is connected to a server he will start the sync process on every commit. As
synclib2's architecture is distributed a server could itself be a client
who is connected to other servers.\\To the latest commit on a database
we refer to as the `head'.

Synchronization follows the following protocol:

\begin{verbatim}
Client has committed to its local database.
Client pushs all commits since the last synced commit to Server.
Client asks Server for the common ancestor of client's head and the server's head
Client pushs all changed data since the common ancestor to Server.

if common ancestor == server head
  // there is no data to merge
  try fast-forward of server's head to client's head
  if failed (someone else updated server's head in the meantime) then start over
else
  Client asks Server for all commits + data since the common ancestor
  Client does a local merge and commits it to the local database
  start over
\end{verbatim}

This protocol is able to minimize the amount of data sent between synced
stores even in a distributed, peer-to-peer setting.

Updating the server's head uses optimistic locking. To update the head
you need to include the last read head in your request.

\section{Handling Conflicts}

\section{Integration with Application Logic}
- demonstrate how to interface with standard MVC frameworks like Backbone, Ember.js

\section{(Managing Changes to Distributed Logic)}
The additional client logic has to be maintained and upgraded for new releases of the application. As the client logic is distributed among all users of the application, a code upgrade becomes more complex to manage than a simple server update. We will see how the same logic used to synchronize application data can be used for updating distributed application code.

- on the server its easy - we can use a distributed version control system
- they don't run on the client -> we need an app-embedded solution




\chapter{Evaluation}\label{evaluation}

We evaluate the implementation based on the set of requirements specified in section~\ref{sec:requirements}.

Evaluate the proof-of-concept by simulating syncing of data structures
used in the problem scenarios with realistic network latency and
disconnection.

show efficiency both on client-server and peer2peer.

implement same Task Manager with different sync backends

use common web framework like Ember.js or Sencha

make a mobile app

evaluate code complexity, robustness, performance

\section{Task Manager using CouchDB}

\begin{itemize}
\item
  Data structure: key-value
\item
  Merging: tree-based
\item
  Propagation: stream-based
\item
  Supports peer-to-peer
\end{itemize}

describe implementation

\section{Task Manager using Histo}

\section{Other Backends}
- Sencha IO (http://www.sencha.com/products/io/) - comes with sync code

- parse.com

- stackmob

- deployd

Most of them simply expose a REST-API but leave all the conflict handling work to the app developer.\\
Are completely centralized.

% Alternative: put content in separate files
% Check the difference between including these files using \input{filename} and \include{filename} and see which one you like better
%\chapter{Einleitung}\label{intro}
%
\chapter{Introduction}
\label{sec:intro}

\section{Motivation}
Applications that allow users to collaborate on data on a central server are in widespread use.
Popular examples are document authoring tools like Google Docs, project collaboration apps like Basecamp or Trello or even large scale collaboration projects like Wikipedia.\\

The traditional architecture of collaborative applications follows a client-server model where the server hosts the entire application logic and persistence.
Users access the application through a thin client, most commonly a web browser.
The browser only has to display user interfaces that are pre-rendered by the server.\\
This model works well when using desktop computers with a realiable, high-speed connection to the server.\\

Rising expectations on the user experience drove developers to increasingly move application logic to the client.
Initially this has only been the logic required to render user interfaces.
The server still hosted most of the application logic to pre-compute all relevant data for the client.\\
Moving the interface rendering to the client reduces the amount of data that has to be transferred and makes the application behave more responsive.\\

The widespread adoption of mobile devices forces developers to re-think their architecture again.
Users can now carry their devices with them and expect their applications to work outside their home or office network.
Applications therefore have to work with limited mobile Internet access or often no access at all.\\
The only way to support this is by moving more of the application logic to the client and by replicating data for offline use.
The clients are now not only responsible for rendering interfaces but also implement most of the application logic themselves.\\
The new architecture comes at a high price - the additional client logic and persistence adds a lot of complexity.
While in the server-centric model developers only had to maintain a single technology set, they now face different technologies on each platform they aim to support with a fat client.\\
The ability to use the application offline requires an entire new layer of application logic to manage the propagation and merging of changes and to resolve conflicts.
The only responsiblity of the server in this model is the propagation of data between clients.\\

Most users today carry a notebook, a smartphone and maybe even a tablet computer with them.
They often want to work with the same data on different devices.
Apps need to support workflows like adding some items to a Todo-Manager on a notebook and subsequently reviewing them on a smartphone.
This implies that even simple applications that are meant for single-users have to aquire collaborative features.
A single-user with multiple devices is from a technical perspective effectively collaborating with itself.\\
Today's applications only achieve this through  data synchronization between the devices and a central server.
If the user is mobile and does not have a reliable Internet connection he is stuck with outdated data on his smartphone.
This problem can only be resolved by supporting the direct synchronization between devices.
The clients can now basically act as servers themselves and manage propagation of data to other clients.\\
The actual server does not have to disappear in this model.
But like the clients it is just another node on the network.
The difference is that the server node is continuously connected to the Internet and can therefore play a useful role as a fallback.\\
Note that this only describes the most extreme scenario - in most real-world applications we will see a hybrid-architecture where clients can synchronize most data directly but the server still manages security or enforces other constraints.\\
Building such a distributed data synchronization engine including all relevant aspects is very complex and beyond the reach of a small team of app developers.
It is also way beyond the scope of this thesis.
As described in the next section we will focus on a set of problem statements and use cases.

TODO:

- add Things app story on how hard it is

- add graphics

\section{Objectives and Approach}

This thesis aims to develop patterns and tools to make the development of offline capable, collaborative apps more productive.\\

The guiding questions are:
\begin{itemize}
\item \emph{Offline Availability}: How can we enable the operation of a collaborative app with frequent network partition?
\item \emph{Synchronization Protocol}: How can we efficiently synchronize changed data directly between unreliably connected devices?
\item \emph{Application Integration}: How can we abstract the synchronization logic to be as unintrusive as possible to an application?\\
\end{itemize}

A collaborative app that has to function with unreliable network connection implies that we can not rely on the traditional thin client model.
We have to think about ways to make both data and logic available offline.\\

Being able to synchronize data directly between devices forces us to develop a distributed architecture.\\

Efficient synchronization means that we aim to minimize the amount of redundant data sent between devices.
We have to figure out ways to identify changes in the data.\\

Combined with the requirement to be unintrusive we exclude solutions that require the application to explicitly track changes in the code.
The identification of data changes should be decoupled from the main application logic.
This ensures that an upgrade of traditional applications requires minimal effort.\\

We will refine this set of requirements by breaking down common use cases and evaluating existing solutions that support offline-capable applications.\\

Important questions which are out of scope of this thesis are:

\begin{itemize}
\item \emph{Security}: How can we manage access rights and encryption in a distributed architecture?
\item \emph{Device Discovery}: How can we discover devices in a network to collaborate with?
\item \emph{Data Transmission}: How is the data propagated among devices on a technical level?
\end{itemize}

TODO:

- phrase problem statements - not only questions

- what is unique to our approach?

\section{Structure of the thesis}
Here you describe the structure of the thesis. For example:

In Kapitel~\ref{background} werden grundlegende Methoden für diese Arbeit vorgestellt.
%
%\chapter{Voraussetzungen}\label{bg}
%
\chapter{Background}
\label{sec:background}
We will start this chapter by explaining the core aspects of data synchronization.
After setting it in context with traditional properties of distributed databases we will present some of the popular approaches to synchronization.\\
Some technical background on the practicality of local data storage based on HTML5 will follow.\\
This will give us a solid background to develop and reason about our own data synchronization framework.

\section{Defining Data Synchronization}
Let us introduce some basic terminology and try to systematically define what data synchronization is about.

\emph{Atoms} are what we define as the lowest level of data that cannot be devided into smaller parts.
Every application may have a custom definition of atoms.
For a file synchronizer it may be entire files, for a source code management system it may be lines in a file, for a collaborative task manager it may be literal values like strings, numbers or dates.\\
Atoms can be aggregated to \emph{objects}.
A source code management system may define objects as a sequence of lines aggregated to a file.
The task manager could aggregate values like strings and dates to task objects.
Objects can themselves be aggregated further into larger objects.
File objects can be combined to directory objects, task objects into larger structures like a project.
Aggregation of objects can also be seens as relationships between objects.\\
A collaborative application has multiple users working on different devices on a related set of data.
They are either connected directly or via servers who live on the local network or the Internet.
Each device, be it a user's device or a server, we define as a \emph{node}.
Nodes can be connected through various network topologies like peer-to-peer, client-server or a hierarchical architecture.
In section \ref{sec:main.requirements.topologies} we will go into more detail about different network topologies.\\
The nodes of mobile users are likely to be partitioned from their network and therefore have to be able to work in \emph{offline} mode.
Therefore application data has to be available locally so that users are not blocked from using their application.
Even when connected to a network it can be beneficial to maintain data locally to increase the responseness of the application.
In a collaborative application, edits that are made locally will have to be \emph{synchronized} with other participating nodes.\\
The process of synchronization can be divided into three phases.
Local edits first have to be identified before they can be sent to other nodes.
We refer to this step as the \emph{update detection} phase.
Some applications may explicitly track each edit as its made and store the history of edit operations.
This \emph{edit-based} approach is necessary for \emph{stream-based synchronization} which we explain in section \ref{sec:background.stream-based}.\\
If edits are not tracked directly we have to run a differencing algorithm to detect updates.
This requires us to keep previous states of the data and is detailed further in section \ref{sec:background.history-based} on \emph{history-based synchronization}.\\
Once updates are detected we continue with the \emph{update propagation} phase.
A stream of edit operations or the differencing output is sent to the collaborating nodes.
The details will be explained in the respective sections on the stream or history based approaches.\\
In a final phase the received data has to be \emph{reconceiled} with the local data on each node.
Updates have to be merged and conflicts are identified.
In a centralized scenario this part is usually carried out by the server.
Distributed architectures supporting peer-to-peer synchronization are much more complex as all clients have to reconcile the received updates in an eventually consistent way.\\

Summarizing these terms:

\begin{itemize}
\item \textbf{Atoms} are the literal values that can not be divided further.
\item \textbf{Objects} aggregate atoms or other objects into larger structures.
\item \textbf{Nodes} are the collaborating devices in an application.
\item The \textbf{update detection} phase identifies local data changes on each node.
\item During \textbf{update propagation} changed data is sent to collaborating nodes.
\item \textbf{Reconciliation} merges data received from other nodes and identifies conflicts.
\end{itemize}

\emph{Update detection}, \emph{update propagation} and \emph{update reconciliation} combined are what we define as \emph{data synchronization}.

\section{ACID Properties and Transactions}
The nodes of a collaborative, mobile application with replicated data represent a distributed database.
Distributed database systems have been a focus of research for decades.
Traditionally, the incentive to make databases distributed has been to provide fault tolerance, increase read/write throughput or to increase storage capacities.
Mobile applications need data replication to reduce frequent network access resulting in a better user experience.
Traditional distributed databases used to back enterprise applications running entirely in server farms.
Servers are connected through reliable and high-speed networks.
Network partitions are the absolute exception.\\
On mobile devices network partitions or slow connections are the norm.
Users want to work with their notebooks even when not being in an office environment with reliable Internet access.
Mobile networks are still comparatively slow and unreliable.\\
Back in 1981 Jim Gray defined the properties of a reliable transaction system \cite{Gray:1981wi}.
They are referred to as the ACID (Atomicity, Consistency, Isolation, Durability) properties - a term coined by Andreas Reuter and Theo Härder \cite{Reuter1983}.
Consistency is defined as the property ensuring that a database can only transition between valid states.
One way to achieve this is to use locking so that a record can not be edited concurrently.
In an always-connected server environment transactions are measured in seconds - locking of data can therefore be acceptable.
In a mobile setting this is not an option as transactions can easily last days.
A mobile user who wants to edit some data while travelling without network connection should certainly not block all other users from doing their work.
Using locking concepts in such a scenario would not only be a an inconvenience for the users but would actually lead to a high-rate of deadlocks.
As Jim Gray states, the rate of deadlocks goes up with the square of the level of concurrency and the fourth power of the transaction size \cite{Gray3}.\\
A common alternative to locking is \emph{multiversion concurrency-control} where readers can still access the prior version of data being editd by another user.
Distributed databases usually use a two-phase commit protocol to gurantee strong consistency.
Each participant has to agree in order to successfully complete a transaction.
In a mobile setting with long periods of disconnection each commit could take hours or days to be acknowledged by all nodes.
This is further complicated as nodes are often not fixed and can be added or removed from a mobile application at any time.


In a distributed database this property can only 

TODO:

- master-less database, no global coordinater

- offline by default --> no globally serializable commits, two-phase commit no suitable

- atomicity (all or nothing commit) only on local nodes

- only eventual consistency

- isolation: has to be relaxed as we cannot allow locking --> concurrent transactions cannot be serialized, use lowest level of isolation ("Read uncommitted", concurrent commits to local databases will eventually be consistent

- durability: only local durability, global durability can not be guaranteed in a master-less database

\section{Stream-Based Synchronization}
\label{sec:background.stream-based}
An application that tracks each edit and sends it in a stream to remote nodes follows a stream-based synchronization protocol.
Stream-based synchronization is very common among real-time document editors like Google Docs.\\

An edit usually represents an insert or delete operation at a certain position in the text.
These edit operations are broadcast to remote nodes and then ``replayed''.
As participating nodes can concurrently edit a document the stream of edit operations can not just be applied without modifications.\\
The combination of local modifications and received edit operations from a remote node requires the transformation of the remote operations in order to be correctly applied.\\
The family of algorithms developed to correctly transform the edit operations is described as \emph{Operational Transformation} \cite{Ellis:1998vf}.\\
If some nodes are temporarily offline while continuing to edit, the correct transformation of many concurrent edit operations becomes very complex and error-prone.\\
A practical problem in modern user interfaces is that it is hard to correctly capture all edits made to data.
If a single edit is missed the result is a fork possibly rendering all future update operations as incorrect.
Packet loss due to unreliable network connections have also be taken into account which further complicates the design of a robust algorithm.\\
Research has therefore investigated options for data synchronization that do not require Operational Transformation.\\

\emph{Commutative Replicated Data Types} (CRDTs) have emerged as a viable alternative for specific use cases.
A recent study by Shapiro et al. presents a range of data types designed for synchronization without concurrency control \cite{Shapiro:2011wy}.\\
CRDTs are designed in a way that all edit operations commute when applied in \emph{causal order}.
Section \ref{sec:main.requirements.causality} goes into more detail about causal ordering of events.
Due to the restrictions on supported operations on data types, CRDTs are only applicable in a narrow set of scenarios.

TODO:

- more details about OT and CRDTs

\section{History-Based Synchronization}
\label{sec:background.history-based}
Snapshot-based methods work by tracking and relating an application's data state over time.
Instead of sending a sequential stream of raw updates, each client collects additional metadata that allows more complex reasoning about the state of each client.\\
A prominent example is the distributed content tracking system \emph{git} \cite{git} which can resolve the most complex peer-to-peer synchronization scenarios.\\
Git achieves this by storing the entire history of a project's database on each client.
Each edit made to objects in the database is stored as a commit object and related to its ancestors.\\
Through the resulting commit graph each client can identify the exact subset of updates each remote node has to receive in order to be in sync.\\
While it sounds extremely inefficient to store the entire history of a database, git manages to do this in a very efficient way through a \emph{Content Addressable Store} and data compression.
It is not uncommon that the uncompressed form of the current state of a git project is larger than the project's entire history.

\section{Three-Way Merging}
\label{sec:background.merging}
Three-way merging describes the concept for an algorithm that performs a merge operation on two objects based on a common ancestor.\\
Let \emph{A} be the initial state of the object and let \emph{B} and \emph{C} be edited versions of \emph{A}.
The goal is to merge \emph{B} and \emph{C} into a new object \emph{D}.\\
The merge algorithm starts by identifying the differences between \emph{A} and \emph{B} and between \emph{A} and \emph{C}.\\
All \emph{parts} of object \emph{B} that are neither changed in \emph{B} nor in \emph{C} are carried over into \emph{D}.\\
All changes to parts of the object in \emph{B} that have not been changed in \emph{C} are directly accepted and added to \emph{D}.\\
If the same parts are edited both in \emph{B} and \emph{C} we have a merge conflict that needs to be resolved.\\
There is no universal algorithm for resolving conflicts.
Different types of data and applications require different types of conflict resolution strategies.
In many cases conflict resolution can not even be done in an automated way but has to be left to the user of an application.\\
Even the term \emph{three-way merging} only describes a general concept but the actual algorithm will differ based on the type of objects that are merged.
Text files are the most common type of object with lines seen as the \emph{parts}.
The unix program \emph{diff3} implements a three-way merge variant for text files \cite{diff3}.\\
Most modern version control systems implement three-way merging to allow lock-free collaboration on source code.
\emph{Git} applies three-way merging not only for text files but for entire file system trees \cite{git}.\\
With git we have a great example of a hierarchical conflict resolution strategy:

\begin{itemize}
\item If two developers concurrently edit the same directory git tries to resolve this conflict by descending into the directory and looking at individual files.
\item If the developers edited different files git can automatically resolve the conflict by accepting both changes.
\item If the same file was edited concurrently git tries to descend a level deeper by looking at edits made to individual lines.
\item If different lines were edited concurrently it can again resolve the conflict by accepting both changes.
\item Only in the unlikely event that both developers edited the same line git has no way to automatically resolve the conflict.
It will delegate the conflict resolution to the developers who will have to manually merge both changes.
\end{itemize}

Tancred Lindholm designed a three-way merging algorithm for XML-documents.
With the \emph{3DM} tool there is even an implementation available \cite{Lindholm:2001uv}.
As XML supports the expression of a broad range of data types this is probably one of the most generic implementations.

\section{Most Recent Common Ancestor}
\label{background.mrca}
- describe problem with graphs

- describe solution referring to standard algo

\section{Content Adressable Storage}
- copy on write

- simple verification of data, free checksums

- git as example

\section{HTML5 and Offline Applications}
HTML5 specifies a number of client-side storage options. Most are a work in process and still have to be adopted by all browser vendors. IndexedDB is most likely going to be the standard for building offline-capable web applications. Combined with Cache Manifests, HTML5 provides all the tools necessary for building offline applications.

\subsection{Web Storage}
The simplest API is the \emph{localStorage} standard defined in the W3C's Web Storage specification \cite{webstorage}.\\
It provides a key-value store accessible from JavaScript which can store string values for string keys.
Most browsers currently set a storage limit of 5 MB per site.
\emph{LocalStorage} is therefore only suitable for storing small volumes of data.\\
Another limitation is the interface which is synchronous. As JavaScript is single-threaded, every read or write operation will block the entire application.
Frequent or large-volume read/write operations can result in a bad user experience caused by a ``freezing'' user-interface.\\
\emph{LocalStorage} is currently supported by all major browsers including its mobile variants.

\subsection{Web SQL Database}
A much more advanced implementation is specified by the now deprecated \emph{Web SQL} standard \cite{websql}. It defines a relational database similar to Sqlite including SQL support.\\
The proposal was strongly opposed by the Mozilla Foundation who sees a SQL-based database as a bad fit for web applications \cite{mozilla_indexeddb}.\\
The standard was therefore only implemented by Google Chrome, Safari and Opera and their mobile counterparts in Android and iOS.\\
\emph{Web SQL} has been officially deprecated by the W3C and support by browsers is likely going to drop in the future.

\subsection{Indexed Database}
Instead of Web SQL the standard favored by the W3C and most browser vendors is \emph{IndexedDB}.\\
\emph{IndexedDB} defines a lower-level interface for storing key/value pairs and setting up custom indexes.
While relatively simple, the API design is generic enough to cater for implementations of more complex databases on top.
It would, for example, be possible to implement a \emph{Web SQL} database using \emph{IndexedDB}.\\
IndexedDB supports storing large amounts of data and defines an asynchronous API.\\
Unfortunately the standard has not yet been implemented across all major browsers.
It is currently available in Mozilla Firefox, Google Chrome and Internet Explorer.
Safari support is still missing as well as support in the default Android and iOS browser.\\
Luckily most browsers who have not implemented IndexedDB yet, are still supporting Web SQL.
There is a polyfill available that implements an IndexedDB interface using Web SQL \cite{indexeddb_polyfill}. Application developers can therefore already base their work on the IndexedDB interface while browser vendors are catching up.

\subsection{Cache Manifests}
To truely work offline, an application has to make its static resources available locally as well.
The \emph{cache manifest} defined in the HTML standard gives developers the right tool \cite{cache_manifests}. It allows you to define a local cache of all application resources like HTML, CSS, JavaScript code or other static files.\\
Flexible policies give fine-grained control over which resources should be available offline and which need network connection.



% References (Literaturverzeichnis):
% a) Style (with abbreviations: use alpha):
\bibliographystyle{unsrt}
% b) The File:
\bibliography{references,manual_references}

\end{document}
