
\chapter{Background}\label{background}

Here you discuss some basics for your work and outline existing research in the area of your thesis by citing research papers like~\cite{Lindholm:2009wo} by Lindholm and~\cite{DeCandia:2007ui,Ratner:2001wz} Candia.

\section{Edit-Based Synchronization}
- application has to track each edit

- concurrent edits have to be transformed

- Operational Transformation
  --> complex and error prone

- Commutative Replicated Data Types
  --> data types that only allow changes that commute
  --> no transformation necessary

\section{State-Based Synchronization}
- async through diff computation

- easier to integrate

- Fraser2009 differential sync

\section{Three-Way Merging}
- 3DM tool (Lindholm)

\section{Most Recent Common Ancestor}
- describe problem with graphs

- describe solution referring to standard algo

\section{HTML5 and Offline Applications}
HTML5 specifies a number of client-side storage options. Most are a work in process and still have to be adopted by all browser vendors. IndexedDB is most likely going to be the standard for building offline-capable web applications. Combined with Cache Manifests, HTML5 provides all the tools necessary for building offline applications.

\subsection{Web Storage}
The simplest API is the \emph{localStorage} standard defined in the W3C's Web Storage specification \cite{webstorage}.\\
It provides a key-value store accessible from JavaScript which can store string values for string keys.
Most browsers currently set a storage limit of 5 MB per site.
\emph{LocalStorage} is therefore only suitable for storing small volumes of data.\\
Another limitation is the interface which is synchronous. As JavaScript is single-threaded, every read or write operation will block the entire application.
Frequent or large-volume read/write operations can result in a bad user experience caused by a ``freezing'' user-interface.\\
\emph{LocalStorage} is currently supported by all major browsers including its mobile variants.

\subsection{Web SQL Database}
A much more advanced implementation is specified by the now deprecated \emph{Web SQL} standard \cite{websql}. It defines a relational database similar to Sqlite including SQL support.\\
The proposal was strongly opposed by the Mozilla Foundation who sees a SQL-based database as a bad fit for web applications \cite{mozilla_indexeddb}.\\
The standard was therefore only implemented by Google Chrome, Safari and Opera and their mobile counterparts in Android and iOS.\\
\emph{Web SQL} has been officially deprecated by the W3C and support by browsers is likely going to drop in the future.

\subsection{Indexed Database}
Instead of Web SQL the standard favored by the W3C and most browser vendors is \emph{IndexedDB}.\\
\emph{IndexedDB} defines a lower-level interface for storing key/value pairs and setting up custom indexes.
The While relatively simple, the API design is generic enough to cater for implementations of more complex databases on top.
It would for example be possible to implement a \emph{Web SQL} database using \emph{IndexedDB}.\\
IndexedDB supports storing large amounts of data and defines an asynchronous API.\\
Unfortunately the standard has not yet been implemented across all major browsers.
It is currently available in Mozilla Firefox, Google Chrome and Internet Explorer.
Safari support is still missing as well as support in the default Android and iOS browser.\\
Luckily most browsers who have not implemented IndexedDB yet, are still supporting Web SQL.
There is a polyfill available that implements an IndexedDB interface using Web SQL \cite{indexeddb_polyfill}. Application developers can therefore already base their work on the IndexedDB interface while browser vendors are catching up.

\subsection{Cache Manifests}
To truely work offline, an application has to make its static resources available locally as well.
The \emph{cache manifest} defined in the HTML standard gives you the right tool \cite{cache_manifests}. It allows you to define a local cache of all application resources like HTML, CSS, JavaScript code or other static files.\\
Flexible policies give fine-grained control over which resources should be available offline and which need network connection.
